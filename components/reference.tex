\newpage
\phantomsection
\section*{KẾT LUẬN}
\addcontentsline{toc}{section}{KẾT LUẬN}

\textbf{Tổng kết}\\
Qua quá trình thực hiện đồ án "Hệ thống Quản lý Sinh viên", nhóm đã hoàn thành thành công các mục tiêu đặt ra:

\begin{enumerate}
	\item \textbf{Xây dựng cơ sở dữ liệu hoàn chỉnh:} Thiết kế 10+ bảng dữ liệu với đầy đủ ràng buộc, khóa chính, khóa ngoại đảm bảo tính toàn vẹn dữ liệu

	\item \textbf{Cài đặt các yêu cầu xử lý phức tạp:} Triển khai 5 Stored Procedures, 3 Functions, 3 Triggers, 2 Cursors và 2 Transactions để xử lý nghiệp vụ

	\item \textbf{Quản trị hệ thống chuyên nghiệp:} Thiết lập hệ thống quản lý người dùng với 5 nhóm quyền, chiến lược backup/restore 3-2-1, bảo mật với TDE và Audit

	\item \textbf{Phát triển ứng dụng web:} Xây dựng ứng dụng ASP.NET MVC với 5+ chức năng chính, giao diện responsive và tích hợp đầy đủ với database
\end{enumerate}

\textbf{Kết quả đạt được}\\
\begin{itemize}
	\item Hệ thống có khả năng quản lý hiệu quả thông tin của 150+ sinh viên, 30+ giảng viên
	\item Xử lý 300+ đăng ký học phần mỗi học kỳ
	\item Thời gian phản hồi trung bình < 1 giây cho hầu hết các thao tác
	\item Hệ thống backup tự động đảm bảo RPO 1 giờ và RTO 4 giờ
	\item Giao diện thân thiện, dễ sử dụng trên nhiều thiết bị
\end{itemize}

\textbf{Kinh nghiệm rút ra}\\
\begin{enumerate}
	\item \textbf{Về thiết kế database:} Việc chuẩn hóa dữ liệu và thiết lập ràng buộc ngay từ đầu giúp tránh nhiều lỗi trong quá trình phát triển

	\item \textbf{Về lập trình T-SQL:} Stored procedures và functions giúp tối ưu hiệu suất và tái sử dụng code hiệu quả

	\item \textbf{Về bảo mật:} Phân quyền chi tiết và encryption là yếu tố quan trọng để bảo vệ dữ liệu nhạy cảm

	\item \textbf{Về phát triển ứng dụng:} Mô hình MVC giúp tổ chức code rõ ràng và dễ bảo trì
\end{enumerate}

\textbf{Hạn chế và hướng phát triển}\\

\textbf{Hạn chế:}
\begin{itemize}
	\item Chưa có module quản lý tài chính học phí
	\item Chưa tích hợp hệ thống gửi email/SMS thông báo tự động
	\item Giao diện mobile app chưa được phát triển
	\item Chưa có tính năng import/export dữ liệu hàng loạt từ Excel
\end{itemize}

\textbf{Hướng phát triển:}
\begin{itemize}
	\item Phát triển thêm module quản lý học phí và công nợ
	\item Tích hợp API để kết nối với các hệ thống khác
	\item Xây dựng mobile app cho sinh viên và phụ huynh
	\item Áp dụng Machine Learning để dự đoán kết quả học tập
	\item Triển khai hệ thống trên cloud (Azure/AWS) để tăng khả năng mở rộng
\end{itemize}


\newpage

\phantomsection
\section*{PHỤ LỤC}
\addcontentsline{toc}{section}{PHỤ LỤC}

\textbf{Phụ lục A: Script tạo database đầy đủ}\\
Script SQL đầy đủ để tạo database, bảng, stored procedures, functions, triggers có thể được tải từ:
\begin{itemize}
	\item File: \texttt{QLSV.sql} trong thư mục gốc dự án
	\item Repository: https://github.com/[your-repo]/QuanLySinhVien
\end{itemize}

\textbf{Phụ lục B: Hướng dẫn cài đặt}\\
\begin{enumerate}
	\item \textbf{Yêu cầu hệ thống:}
	      \begin{itemize}
		      \item Windows 10/11 hoặc Windows Server 2019+
		      \item SQL Server 2019 hoặc mới hơn
		      \item .NET Framework 4.8
		      \item IIS 10.0+
		      \item RAM: Tối thiểu 8GB
		      \item Disk: Tối thiểu 20GB trống
	      \end{itemize}

	\item \textbf{Cài đặt database:}
	      \begin{itemize}
		      \item Chạy script \texttt{QLSV.sql} trong SSMS
		      \item Cấu hình login và users theo hướng dẫn Chương 3
		      \item Thiết lập backup jobs
	      \end{itemize}

	\item \textbf{Deploy ứng dụng:}
	      \begin{itemize}
		      \item Build solution trong Visual Studio
		      \item Publish to IIS
		      \item Cấu hình connection string trong Web.config
		      \item Test các chức năng chính
	      \end{itemize}
\end{enumerate}

\textbf{Phụ lục C: Danh sách các stored procedures và functions}\\

\begin{table}[H]
	\centering
	\begin{tabular}{|l|l|p{6cm}|}
		\hline
		\textbf{Loại} & \textbf{Tên}                 & \textbf{Mô tả}                            \\
		\hline
		Procedure     & sp\_ThemSinhVien             & Thêm sinh viên mới với kiểm tra ràng buộc \\
		Procedure     & sp\_DangKyHocPhan            & Đăng ký học phần cho sinh viên            \\
		Procedure     & sp\_TinhDiemTongKet          & Tính điểm tổng kết và xếp loại            \\
		Procedure     & sp\_ThongKeDiemTheoLop       & Thống kê điểm theo lớp                    \\
		Procedure     & sp\_XetHocBong               & Xét học bổng tự động                      \\
		\hline
		Function      & fn\_TinhDiemTrungBinh        & Tính điểm trung bình tích lũy             \\
		Function      & fn\_TongTinChiDaDangKy       & Tính tổng tín chỉ đã đăng ký              \\
		Function      & fn\_KiemTraDieuKienTotNghiep & Kiểm tra điều kiện tốt nghiệp             \\
		\hline
		Trigger       & trg\_CapNhatSiSoLop          & Tự động cập nhật sĩ số lớp                \\
		Trigger       & trg\_KiemTraDiemHopLe        & Kiểm tra và tính điểm tự động             \\
		Trigger       & trg\_LogThayDoiSinhVien      & Ghi log thay đổi thông tin                \\
		\hline
	\end{tabular}
\end{table}

\textbf{Phụ lục D: Mã nguồn quan trọng}\\
Mã nguồn đầy đủ của dự án có thể được truy cập tại:
\begin{itemize}
	\item Thư mục: \texttt{C:\textbackslash Users\textbackslash thanh\textbackslash Downloads\textbackslash BaoCao\textbackslash QLSV\textbackslash QuanLySinhVien}
	\item Cấu trúc thư mục đã được mô tả chi tiết trong Chương 4
\end{itemize}
