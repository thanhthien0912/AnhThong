\newpage
\section{QUẢN TRỊ HỆ THỐNG}

\subsection{Giới thiệu}
Chương này trình bày các khía cạnh quan trọng trong quản trị hệ thống cơ sở dữ liệu SQL Server cho hệ thống QLSV\_DoAn\_2, bao gồm: quản lý người dùng, phân quyền bảo mật, và chiến lược sao lưu/phục hồi dữ liệu. Hệ thống được thiết kế với các bảng chính: HeDaoTao, Khoa, ChucVu, GiangVien, Lop, SinhVien, MonHoc, HocKy, LopHocPhan, DangKyHocPhan, MonHoc\_TienQuyet, Account, LoginLog, và LogoutLog.

\subsection{Quản trị người dùng và phân quyền}

\textbf{Cơ chế xác thực}\\
Hệ thống sử dụng \textbf{Mixed Mode Authentication} kết hợp cả hai phương thức:
\begin{itemize}
	\item \textbf{Windows Authentication:} Cho các quản trị viên hệ thống và nhân viên nội bộ
	\item \textbf{SQL Server Authentication:} Cho ứng dụng web và người dùng từ xa
\end{itemize}

\textbf{Lý do lựa chọn Mixed Mode:}
\begin{enumerate}
	\item Linh hoạt trong việc quản lý người dùng từ nhiều nguồn khác nhau
	\item Ứng dụng web có thể kết nối từ các máy chủ không thuộc domain
	\item Dễ dàng quản lý và cấp quyền cho từng nhóm người dùng
	\item Hỗ trợ cả môi trường phát triển và production
\end{enumerate}

\textbf{Tạo Login và User}\\
\textbf{Login} là tài khoản để xác thực ở cấp SQL Server instance, còn \textbf{User} là tài khoản được map từ Login để truy cập database cụ thể.

\begin{lstlisting}[style=sql, caption={Tạo Login cho các nhóm người dùng}]
-- 1. Tao Login cho sinh vien
CREATE LOGIN sinhvien
WITH PASSWORD = 'sinhvien@123', 
     DEFAULT_DATABASE = QLSV_DoAn_2;

-- 2. Tao Login cho giang vien
CREATE LOGIN giangvien
WITH PASSWORD = 'giangvien@123', 
     DEFAULT_DATABASE = QLSV_DoAn_2;

-- 3. Tao Login cho quan tri (su dung cho Account table)
-- Vi du: admin, baotv, annv da duoc them vao bang Account
-- Mat khau duoc ma hoa SHA256
GO
\end{lstlisting}

\begin{lstlisting}[style=sql, caption={Tạo User trong database}]
USE QLSV_DoAn_2;
GO

-- Tao User cho sinh vien
CREATE USER user_sinhvien
FOR LOGIN sinhvien;

-- Tao User cho giang vien
CREATE USER user_giangvien
FOR LOGIN giangvien;

GO
\end{lstlisting}

\textbf{Tạo Role và phân quyền}\\
Database Roles giúp nhóm các quyền lại và dễ dàng quản lý phân quyền cho nhiều user.

\begin{lstlisting}[style=sql, caption={Tạo các Role trong database}]
USE QLSV_DoAn_2;
GO

-- 1. Tao nhom quyen cho sinh vien
CREATE ROLE role_sinhvien;

-- 2. Tao nhom quyen cho giang vien
CREATE ROLE role_giangvien;

GO
\end{lstlisting}

\begin{lstlisting}[style=sql, caption={Cấp quyền cho các Role}]
-- 1. Sinh vien chi duoc xem du lieu
GRANT SELECT ON dbo.SinhVien TO role_sinhvien;
GRANT SELECT ON dbo.Lop TO role_sinhvien;
GRANT SELECT ON dbo.MonHoc TO role_sinhvien;
GRANT SELECT ON dbo.HocKy TO role_sinhvien;
GRANT SELECT ON dbo.LopHocPhan TO role_sinhvien;
GRANT SELECT ON dbo.DangKyHocPhan TO role_sinhvien;

-- Quyen thuc thi cac function lien quan
GRANT EXECUTE ON fn_TinhDiemTrungBinh TO role_sinhvien;
GRANT EXECUTE ON fn_KiemTraTienQuyet TO role_sinhvien;

-- 2. Giang vien duoc xem va cap nhat du lieu
GRANT SELECT, INSERT, UPDATE, DELETE 
ON dbo.GiangVien 
TO role_giangvien;

GRANT SELECT, INSERT, UPDATE 
ON dbo.DangKyHocPhan 
TO role_giangvien;

GRANT SELECT ON dbo.SinhVien TO role_giangvien;
GRANT SELECT ON dbo.Lop TO role_giangvien;
GRANT SELECT ON dbo.MonHoc TO role_giangvien;
GRANT SELECT ON dbo.LopHocPhan TO role_giangvien;

-- Quyen thuc thi stored procedures
GRANT EXECUTE ON sp_DangKyHocPhan TO role_giangvien;
GRANT EXECUTE ON sp_CapNhat_ChucVu_GV TO role_giangvien;
GRANT EXECUTE ON sp_ThemMonHoc TO role_giangvien;

-- 3. Thu hoi quyen DELETE khoi giang vien
REVOKE DELETE 
ON dbo.GiangVien
FROM role_giangvien;

GO
\end{lstlisting}

\begin{lstlisting}[style=sql, caption={Gán User vào Role}]
-- Them user vao role
ALTER ROLE role_sinhvien ADD MEMBER user_sinhvien;
ALTER ROLE role_giangvien ADD MEMBER user_giangvien;

-- Kiem tra quyen cua nguoi dung
EXEC sp_helprolemember 'role_sinhvien';
EXEC sp_helprolemember 'role_giangvien';
GO
\end{lstlisting}

\textbf{Quản lý tài khoản và bảo mật}\\
Hệ thống sử dụng bảng \textbf{Account} để quản lý tài khoản người dùng với mật khẩu được mã hóa SHA256.

\begin{lstlisting}[style=sql, caption={Tạo bảng quản lý tài khoản}]
-- Tao bang Account
CREATE TABLE Account (
    MaTaiKhoan NVARCHAR(20) PRIMARY KEY,
    TenDangNhap NVARCHAR(50) NOT NULL UNIQUE,
    MatKhau NVARCHAR(MAX) NOT NULL, -- Ma hoa SHA256
    Email VARCHAR(100) NOT NULL UNIQUE,
    LoaiTaiKhoan NVARCHAR(20) 
        CHECK (LoaiTaiKhoan IN ('Admin', 'Lecturer', 'Student')),
    TrangThai BIT DEFAULT 1,
    NgayTao DATETIME DEFAULT GETDATE()
);

-- Tao bang LoginLog de ghi nhat ky dang nhap
CREATE TABLE LoginLog (
    ID INT IDENTITY PRIMARY KEY,
    TenDangNhap NVARCHAR(50),
    NgayGio DATETIME DEFAULT GETDATE(),
    ThanhCong BIT,
    DiaChiIP VARCHAR(20)
);

-- Trigger kiem tra mat khau phai duoc ma hoa
CREATE TRIGGER trg_KiemTraMatKhau_Account
ON Account
FOR INSERT, UPDATE
AS
BEGIN
    IF EXISTS (
        SELECT * FROM inserted
        WHERE LEN(MatKhau) < 40 -- SHA256 toi thieu 40 ky tu
    )
    BEGIN
        RAISERROR(N'Loi: Mat khau phai duoc ma hoa!', 16, 1);
        ROLLBACK TRANSACTION;
    END
END
GO
\end{lstlisting}

\subsection{Sao lưu và phục hồi dữ liệu}

\textbf{Chiến lược sao lưu}\\
Hệ thống áp dụng chiến lược sao lưu 3-2-1:
\begin{itemize}
	\item \textbf{3 bản sao:} Dữ liệu gốc + 2 bản backup
	\item \textbf{2 phương tiện khác nhau:} Ổ cứng local và cloud storage
	\item \textbf{1 bản offsite:} Lưu trữ từ xa để phòng thảm họa
\end{itemize}

\begin{table}[H]
	\centering
	\caption{Lịch trình sao lưu}
	\begin{tabular}{|l|l|l|l|}
		\hline
		\textbf{Loại backup} & \textbf{Tần suất} & \textbf{Thời gian}     & \textbf{Lưu trữ} \\
		\hline
		Full Backup          & Hàng tuần         & Chủ nhật 00:00         & 4 tuần           \\
		Differential Backup  & Hàng ngày         & 00:00                  & 7 ngày           \\
		Transaction Log      & 4 giờ/lần         & 00:00, 04:00, 08:00... & 3 ngày           \\
		\hline
	\end{tabular}
\end{table}

\textbf{Script sao lưu thủ công}\\

\begin{lstlisting}[style=sql, caption={Script Full Backup}]
-- Sao luu toan bo database
BACKUP DATABASE QLSV_DoAn_2
TO DISK = 'D:\Do an HQTCSDL.bak'
WITH INIT,        
     NAME = 'Full Backup QLSV_DoAn',
     SKIP,
     FORMAT,
     STATS = 10;
GO

-- Verify backup
RESTORE VERIFYONLY 
FROM DISK = 'D:\Do an HQTCSDL.bak';
GO
\end{lstlisting}

\begin{lstlisting}[style=sql, caption={Script Differential Backup}]
-- Sao luu khac biet (chi sao luu phan thay doi)
BACKUP DATABASE QLSV_DoAn_2
TO DISK = 'D:\Do an HQTCSDL_Diff_' 
          + CONVERT(VARCHAR(8), GETDATE(), 112) + '.bak'
WITH DIFFERENTIAL,
     NAME = 'Sao luu khac biet QLSV_DoAn',
     STATS = 10;
GO
\end{lstlisting}

\begin{lstlisting}[style=sql, caption={Script Transaction Log Backup}]
-- Thiet lap recovery mode
ALTER DATABASE QLSV_DoAn_2
SET RECOVERY FULL;
GO

-- Sao luu nhat ky giao dich
BACKUP LOG QLSV_DoAn_2
TO DISK = 'D:\QLSV_DoAn_Log_' 
          + CONVERT(VARCHAR(8), GETDATE(), 112) + '.trn'
WITH INIT,
     NAME = 'Sao luu nhat ky giao dich QLSV_DoAn';
GO
\end{lstlisting}

\textbf{Script sao lưu tự động}\\

\begin{lstlisting}[style=sql, caption={Script tự động sao lưu với ngày tháng}]
-- Script sao luu tu dong voi timestamp
DECLARE @BackupFile NVARCHAR(255)
SET @BackupFile = 'D:\QLSV_DoAn_' 
    + CONVERT(VARCHAR(8), GETDATE(), 112) + '.bak'

BACKUP DATABASE [QLSV_DoAn_2]
TO DISK = @BackupFile
WITH INIT, STATS = 10;
GO

-- Script kiem tra backup history
SELECT 
    database_name,
    backup_start_date,
    backup_finish_date, 
    type,
    backup_size/1024/1024 AS backup_size_MB
FROM msdb.dbo.backupset
WHERE database_name = 'QLSV_DoAn_2'
ORDER BY backup_start_date DESC;
\end{lstlisting}

\textbf{Phục hồi dữ liệu}\\

\begin{lstlisting}[style=sql, caption={Script phục hồi database}]
-- 1. Phuc hoi tu Full Backup
RESTORE DATABASE QLSV_DoAn_2 
FROM DISK = 'D:\Do an HQTCSDL.bak'
WITH FILE = 1,
     REPLACE,
     STATS = 10;
GO

-- 2. Neu co Differential Backup
RESTORE DATABASE QLSV_DoAn_2 
FROM DISK = 'D:\Do an HQTCSDL_Diff_20250101.bak'
WITH FILE = 1,
     NORECOVERY,
     STATS = 10;
GO

-- 3. Ap dung Transaction Log neu co
RESTORE LOG QLSV_DoAn_2 
FROM DISK = 'D:\QLSV_DoAn_Log_20250101.trn'
WITH FILE = 1,
     RECOVERY;
GO

-- 4. Kiem tra database sau khi phuc hoi
DBCC CHECKDB('QLSV_DoAn_2') WITH NO_INFOMSGS;
GO
\end{lstlisting}


\subsection{Các Stored Procedures, Functions và Triggers}

\textbf{Stored Procedures quan trọng}\\

\begin{lstlisting}[style=sql, caption={Procedure đăng ký học phần cho sinh viên}]
CREATE PROCEDURE sp_DangKyHocPhan
    @MaSV NVARCHAR(10),
    @MaLHP NVARCHAR(10)
AS
BEGIN
    SET NOCOUNT ON;

    -- Kiem tra lop hoc phan ton tai
    IF NOT EXISTS (SELECT 1 FROM LopHocPhan WHERE MaLHP = @MaLHP)
    BEGIN
        PRINT N'Lop hoc phan khong ton tai!';
        RETURN;
    END;

    -- Kiem tra sinh vien da dang ky
    IF EXISTS (SELECT 1 FROM DangKyHocPhan 
               WHERE MaSV = @MaSV AND MaLHP = @MaLHP)
    BEGIN
        PRINT N'Sinh vien da dang ky lop nay!';
        RETURN;
    END;

    -- Kiem tra so luong
    DECLARE @SL INT = (SELECT COUNT(*) FROM DangKyHocPhan 
                       WHERE MaLHP = @MaLHP);
    DECLARE @ToiDa INT = (SELECT SoLuongToiDa FROM LopHocPhan 
                          WHERE MaLHP = @MaLHP);

    IF @SL >= @ToiDa
    BEGIN
        PRINT N'Lop hoc phan da day!';
        RETURN;
    END;

    -- Dang ky
    INSERT INTO DangKyHocPhan (MaSV, MaLHP)
    VALUES (@MaSV, @MaLHP);

    PRINT N'Dang ky thanh cong!';
END;
GO
\end{lstlisting}

\textbf{Functions quan trọng}\\

\begin{lstlisting}[style=sql, caption={Function tính điểm trung bình của sinh viên}]
CREATE FUNCTION fn_TinhDiemTrungBinh (@MaSV NVARCHAR(10))
RETURNS FLOAT
AS
BEGIN
    DECLARE @DTB FLOAT;
    
    SELECT @DTB = AVG(DiemTongKet)
    FROM DangKyHocPhan
    WHERE MaSV = @MaSV
      AND DiemTongKet IS NOT NULL;
    
    RETURN @DTB;
END;
GO
\end{lstlisting}

\textbf{Triggers quan trọng}\\

\begin{lstlisting}[style=sql, caption={Trigger tự động tính điểm tổng kết}]
CREATE TRIGGER trg_TinhDiemTongKet
ON DangKyHocPhan
AFTER INSERT, UPDATE
AS
BEGIN
    SET NOCOUNT ON;
    
    UPDATE D
    SET D.DiemTongKet = ROUND(
        (ISNULL(D.DiemChuyenCan, 0)*0.1 +
         ISNULL(D.DiemGiuaKy, 0)*0.3 +
         ISNULL(D.DiemCuoiKy, 0)*0.6), 2)
    FROM DangKyHocPhan D
    JOIN inserted I
        ON D.MaSV = I.MaSV AND D.MaLHP = I.MaLHP;
END;
GO
\end{lstlisting}

\begin{lstlisting}[style=sql, caption={Trigger ngăn xóa chức vụ đang sử dụng}]
CREATE TRIGGER TR_NGAN_CHAN_XOA_CHUCVU
ON ChucVu
INSTEAD OF DELETE
AS
BEGIN
    SET NOCOUNT ON;
    
    IF EXISTS (
        SELECT 1
        FROM deleted d
        INNER JOIN GiangVien g ON d.MaChucVu = g.MaChucVu
    )
    BEGIN
        RAISERROR(N'Loi: Khong the xoa Chuc vu. 
            Van con Giang vien duoc gan voi Chuc vu nay.', 16, 1);
        RETURN;
    END
    ELSE
    BEGIN
        DELETE FROM ChucVu
        WHERE MaChucVu IN (SELECT MaChucVu FROM deleted);
    END
END
GO
\end{lstlisting}

\newpage
